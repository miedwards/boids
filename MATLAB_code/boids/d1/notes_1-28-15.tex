\documentclass{article}
\usepackage{amsmath}
\usepackage{amssymb}
\usepackage{mathrsfs}
\usepackage{tikz}
\usepackage{graphicx}
\usepackage{subcaption}

\title{Notes for Autonomous Robots}
\author{Professor J. Issacs}
\date{1/28/15}

\begin{document}
\maketitle
\section{Introduction}
\begin{itemize}
\item Starling mermeration $\rightarrow$ an example of a network of autonomous systems
\item Sardines 
\end{itemize}
This inspires the following questions.
\begin{itemize}
\item Who is the leader?
\item Are there multiple leaders
\item Is there a leader at all?
\item How do they communicate
\item Who communicates to whom
\item Do they use sensors
\item How much of their time is spent concentrating on how to move. 
\end{itemize}
These animals are able to
\begin{itemize}
\item Deploy over a given region
\item Assume a specified pattern
\item Rendezvous at a common point
\item Jointly initiate motion/change direction in a synchronized way.
\end{itemize}
Each individual
\begin{itemize}
\item Senses its immediate environment
\item Communicates with the others.
\item Processes information gathered
\item Takes local action
\end{itemize}

\section{Modelling Physical Motion}
\begin{itemize}
\item Linear Motion -- a single integrator 
\item The Norm is the Length 
\item Linear motion -- a double integrator
\item $x \in \mathbb{R}^4, u \in \mathbb{R}^2$ where $x$ is both position and velocity.
\[ \dot{x} = \]
\item Nonlinear models
\item Unicycle 
\[ x \in \mathbb{R}^3, u \in \mathbb{R}^2 \]
\[ x = \begin{bmatrix} x \\ y \\ \theta \end{bmatrix}, 
u  = \begin{bmatrix} v \\ u \end{bmatrix} \]
Where $v$ is linear speed and $u$ is angular speed.

\item Dubins Vehicle 
\[ \dot{x} = \begin{bmatrix} v\cos \theta \\ v \sin \theta \\ u \end{bmatrix} \]
Where $v$ is constant and $u$ is bounded. 

\item Differential drive robot. 
\item A wheeled mobile robot - a lot like the turtle in turtle graphics. 
\item The Dirty Derivative
\[ f'(x) = \text{lim}_{h\rightarrow 0 }\frac{f(x+h) -f(x)}{h} \]
Which we can approximate to low fidelity. Now, for major work we'd need better
methods. 
\item Retuning to Dubins
\[ \dot{x} = \begin{bmatrix}  v\cos \theta \\ v \sin \theta \\ u \end{bmatrix}
\]
\[ x(t+h) = x(t) + h \begin{bmatrix} v \cos \theta \\ v \sin \theta \\ u
\end{bmatrix} \]

\section*{Reynolds Flocking and Boids Rules}
A basic algorithm for flocking. This has 3 components.
\begin{enumerate}
\item Separation to avoid crowding of local flockmates
\item Alignment to move towards the average heading of local flockmates
\item Cohesion steering toward the average position of local flockmates. 
\end{enumerate}


\end{itemize}



\end{document}
