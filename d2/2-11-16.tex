\documentclass{article}
\usepackage[usenames,dvipsnames]{pstricks}
\usepackage{amsfonts}
\usepackage{amsmath}
\usepackage{amssymb}
\usepackage{epsfig}
\usepackage{graphicx}
\usepackage{mathrsfs}
\usepackage{pst-grad} % For gradients
\usepackage{pst-plot} % For axes
\usepackage{subcaption}
\usepackage{tikz}

\title{Notes for Autonomous Robotic Networks}
\author{Professor Jason Isaacs}
\date{i2/11/16}

\begin{document}
\maketitle
\section{More mathematical fundamentals}
\[ G=(VE) \]
Then 
\[ E \subseteq V \times V \]
Where $G$ is the graph, $V$ is the vertex set and $E$ is the edge set. 
There are classes of graphs. 
\begin{itemize}
\item A ``Complete Graph'' is a completely connected graph, where each node is
connected to every other node. 
\item A ``Path Graph'' $P_n$ for $n$-verticies is a graph comprised of a path. A
path graph can be any graph isomorphic to the graph
\[ P_n = (\{v_1, v_2,\ldots,v_3\},E_p) \]
where 
\[ (v_i, v_j) \in E_p \text{ iff } j = i+1, i = 1,2\ldots, n-1 \]
That is, it is the ``connectivity opposite'' of a complete graph that is still
fully connected.

\item A Cycle graph is a graph that contains a single cycle through all the
nodes. 
\[ C_n = \{V, E \}, (v_i,v_j)\in E_c \text{ iff } |i-j| = \pm 1 \mod n \]

\item A ``Star Graph'' $S$ is a with one internal node, that is connected to all
other nodes. 

\[ S_n = (V,E_s) \text{ where each edge connects to the ``center node'' } \]

\item A regular graph, is defined by the following property. Each vertex of a $k$-regular has degree $k$. 

\item Bipartite Graph. The vertex set is the union of two disjoin vertex sets
$V_1$ and $V_2$. 
If 
\[ |V_1| = m, |V_2| = n \]
\[ V(G) = V_1 \cup V_2 \]
with $nn$ edges called complete bipartite graph. 

\item A Graph $G$ is connected if for every pair of verticies in $V(G)$ there is
a path that has them as endpoints. 

\end{itemize}

In class assignment. Create a program that checks connectivity

Spectral Graph Theory uses eigenvalues and eigenvectors to analyze graphs. The
Graph Laplacian Matrix is positive semi-definite, has real eigenvalues can be
ordered
\[ \lambda_1(G) \leq \lambda_2(G) \cdots \ldots \leq \lambda_n(G) \]
Theorem, the graph $G$ is connected iff $\lambda_2(G) > 0$. 

\section*{Agreement/Consesnsus (Chapter 3)}
\[ \dot{x}(t) = \sum_{j \in N(i)} \left( x_j(t) - x_i(t) \right),
i=1,2,3\ldots,n \]
Where $N(i)$ are the neighbors of $i$. Another representation of this is thus
\[ \dot{x}(t) = -L(G)x(t) \]
for graphs with 5 nodes.
We can now answer these problems from the homework. Compare the rate of
convergence as the number of edges increases. Does this always improve when more
edges are added. Why, or why not? Try to find the settling time for every graph
we generate within some tolerance such as 2\%. Also, the rate of convergence
stems from the eigenvalues. 

also consider the \texttt{expm} function and note that the ODE
\[ \dot{x} = a x \]
has the solution
\[ x(t) = x(0) e^{at} \]
and so we can also find this mathematically. Also, look at blackboard. And ask
questions and the professor will add papers for projects.

\end{document}
