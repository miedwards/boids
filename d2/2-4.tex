\documentclass{article}
\usepackage[usenames,dvipsnames]{pstricks}
\usepackage{amsfonts}
\usepackage{amsmath}
\usepackage{amssymb}
\usepackage{epsfig}
\usepackage{graphicx}
\usepackage{mathrsfs}
\usepackage{pst-grad} % For gradients
\usepackage{pst-plot} % For axes
\usepackage{subcaption}
\usepackage{tikz}

\title{Notes for Autonomous Robotic Networks}
\author{Professor Jason Isaacs}
\date{}

\begin{document}
\maketitle
\section{Introduction}
Professor Isaacs is from Kentucky, and is the first member of his family to go
to college. Got his Bacchelors in Physics, and then worked at a Control Systems
Engineering company? He was interested for 2 years, but the lack of innovation
caused him to go to UCSB to get his PhD in Control Systems. 

Andrew Trough --- his company Chessapeke Technology, a military contract corp,  is
paying for his Masters. He was an undergrad here and grew up in this area. 

Matt --- Bachelors in Physics in UCSD --- didn't like it because he was

Dharini

Dharuv --- wanted to be a professor. 

Sing --- animation, ee degree. 

Vishanta --- born in Shri Lanka --- was an EE in central Africa. Moved to the US
16--17 years ago. 

\section{Graph Theory}
We will be designing systems for groups of agents who have only local
information. This information includes:
\begin{enumerate}
\item Communication
\item Sensing
\end{enumerate}
Note that connections are not always circular over a distance, and are generally
not associative. This defines the graphs that we use in our model. Generally,
all to all communication is impossible. Connections may be directed, but we
won't worry about that now. 

\subsection{Sensing}
One example of sensing is a video feed. This is directional connection, but we
won't discuss this too much. Other examples include
\begin{itemize}
\item Omnidirectional laser rangefinder. 
\item Tactile sensor.
\item A Single ray. 
\end{itemize}

This motivates these graph models. 
\begin{itemize}
\item Nodes (verticies)
\item Edges --- connections. 
\end{itemize}

Three Network Graphs
\begin{enumerate}
\item Static Network --- If $x_1$ can communicate with $x_2$ they will always be
able to communicate. This can be studied using Linear Systems (LTI)


\item Dynamic Networks --- for mobile agents. If $x_1$ can communicate with $x_2$,
this may be state dependent and time variant. Hybrid System (Non-linear
stability)

\item Random Networks --- from Packet Drops. Lyapunor Theory or Stochastic
Stability.
\end{enumerate}

We will cover
\begin{enumerate}
\item Consensus --- global agreement among the agents. Given only local
communication, can the entire group converge into a common value. 
\item Formations. Can we do this with algorithms. 
\item Assignments --- task allocation. How do we assign roles to individuals
autonomously. Concord algorithm --- for the Traveling Salesman Problem (TSP) which is quite popular challenge
for assignments. 
\item Coverage. 
\item Flocking / Swarming. 
\item Social Networks. This is good for finding opinion dynamics. 
\item Distributed Estimation. This is similar to consensus, but can be more
general. It can mean more than averaging. 
\end{enumerate}

\section{Graph Theory (for real this time)}
Graphs are made up of vertices and edges. Let $V$ be the vertices and $E$ be the
edges, and the graph then be $G(V,E)$. We now cover undirected graphs. We define
the neighborhood $N(i)$ of $i$ is defined as 
\[ N(i) \equiv \{v_j \in V | (v_i, v_j) \in E\} \subseteq V \]
Now, if $v_j \in N(i)$ then $v_i \in N(j)$. Then a path through the graph goes
across edges. Similarly, a $m$-length path is exactly what you'd expect. We now
define end points and interior points. If all vertices in the sequence are
distinct, except the end points, then this is a cycle. It appears a path may
contain a cycle, but it is not generally used that way. 

Connectivity: a graph $G$ is connected if for every pair of vertices $V(G)$
there is a path with them as end points. 

Degree: for an undirected graph $G$, the degree for a given vertex $D(v_i)$ is
the cardinality of the neighborhood set $N(i)$. 

We will now define the degree matrix $\Delta(G)$ which is a diagonal matrix of
the degrees. 
\[ \Delta(G) = \begin{bmatrix} d(v_1) & 0 & \cdots & 0 \\
0 & d(v_2) & \cdots & 0 \\
0 & 0 & \cdots & d(v_i) \end{bmatrix} \]
An $n \times n$ Adjacency Matrix is defined as 
\[ {[A(G)]}_{ij} = 
    \begin{cases} 1 &\text{if}(v_i, v_j) \in E \\
        0 &\text{otherwise}
            \end{cases} 
\]
The graph Laplacian is defined as
\[ L(G) = \Delta(G) - A(G) \]



\end{document}
