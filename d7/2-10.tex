\documentclass{article}
\usepackage[usenames,dvipsnames]{pstricks}
\usepackage{amsfonts}
\usepackage{amsmath}
\usepackage{amssymb}
\usepackage{epsfig}
\usepackage{graphicx}
\usepackage{mathrsfs}
\usepackage{pst-grad} % For gradients
\usepackage{pst-plot} % For axes
\usepackage{subcaption}
\usepackage{tikz}

\title{Notes for Autonomous Robotic Networks}
\author{Professor Jason Isaacs}
\date{2/10/16}

\begin{document}
\maketitle
\section{Distribuited Architecture}
A Synchronous network $S$ is a digraph $S=(I,E_{con})$, $I = \{1,2,\ldots,n\}$
\[ (i,j)\in E_{con} \rightarrow i \text{ can send messages to } j \]

A distruibted algorithm $DA$ for network $S$ has an alphabet $\mathbb{A}$,
processor state $W^{[i]}, i \in I$, and allowable state states. See section 1.5
of BulloCortesMartinez. Read this, instead of notes. 

\section{Network Evolution}
Given an intial state, 
\[  w_0^{[i]} \in W_0^{[i]}, i \in I \]
\[ q^{[i]}(l) = \text{stf}^{[i]}(w^{[i]}(l), y^{[i]}(l)) \]
Where $l$ is time and 
\[ y^{[i]}(l)  = \left\{ \begin{array} \text{msg}^{[j]}(w^{[j]}(l-1),i), &(j,i)
\in E_{con} \\
null, &else \end{array} \right. \]

\end{document}
