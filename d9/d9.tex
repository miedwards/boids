\documentclass{article}
\usepackage[usenames,dvipsnames]{pstricks}
\usepackage{amsfonts}
\usepackage{amsmath}
\usepackage{amssymb}
\usepackage{epsfig}
\usepackage{graphicx}
\usepackage{mathrsfs}
\usepackage{pst-grad} % For gradients
\usepackage{pst-plot} % For axes
\usepackage{subcaption}
\usepackage{tikz}

\title{Notes for Autonomous Robotic Networks}
\author{Professor Jason Isaacs}
\date{4/21/16}

\begin{document}
\maketitle
\section{The Final Presentation (Taken from the Seminar)}
The presentation should last 20 minutes with questions. Be ready to go from the
get go. Have everything prepared in advance. We ready to walk in and start
talking. Have it on your laptop, your thumb drive, online, in PDF and normal
format. Remember that your talk is not a thesis defense, a job interview, a
classroom lecture, or some sort of paper meant to be read later. A good pace is
about 1-2 minutes per slide. Remember that you should have about 1 slide on the
motivation for your research. 

The title slide should have the title and. Your opening statement should be
planned. Include Thank the person who introduces you, introduce yourself and
thank the audience of their attendance, acknowledge your school, collaborators,
and give a short transition into your next slide. 

\section{Distributed Algorithms}
This will be the last of the Bullo book for us. The Linear Combination algorithm
has an alphabet $\mathbb{A} = \mathbb{R} \cup \text{null}$ and a process state
$w \in \mathbb{R}$. See 1.6.1 from Bullo. 

\section{Some Interesting Problems}
In general $n$-bugs will trace out logarithmic patterns that converge at a
single point. This is the rendesvous problem. A more recent variant has
applications to boundary guarding. 




\end{document}
