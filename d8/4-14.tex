\documentclass{article}
\usepackage[usenames,dvipsnames]{pstricks}
\usepackage{amsfonts}
\usepackage{amsmath}
\usepackage{amssymb}
\usepackage{epsfig}
\usepackage{graphicx}
\usepackage{mathrsfs}
\usepackage{pst-grad} % For gradients
\usepackage{pst-plot} % For axes
\usepackage{subcaption}
\usepackage{tikz}

\title{Notes for Autonomous Robotic Networks}
\author{Professor Jason Isaacs}
\date{4/14/16}

\begin{document}
\maketitle
\section{Centralized Algorithms}
From 1.4.4 and 2.4.4 of Bullo. Given a digraph $G$ we say that the length of a
directed path is the number of edges in a path. Given verticies $u$ and $v$ in
$G$ the distance from $u$ to $v$ is the length of the shortest path between
them. This is sometimes written as 
\[ \text{dist}_{G}(u,v)  = \text{min}(\{\text{length}(p)|p \text{ is a directed
path from }v\text{ to }u) \]
Given a vertex $v$ in $G$ the radius of $v$ in $G$ is the max of all  distances
from $v$ to all other nodes in $G$. 
\[ \text{radius}(v,G) = \text{max}\{\text{dist}_G(v,u)|u\in V(G)\} \]
Let $v$ be a vertex of $G$ with radius of $v$ in $G$ less than positive
infinity. Then $G$ is connected, since there is a path between all all vertexs
$G$. A Breadth first spanning tree (BFS) is a spanning tree made using BFS\@. 

For a digragph $G$ with respect to vertex $v$, then the tree $T$ prduced by
performing BFS starting from $v$ contains the shortest path from $v$ to all
other reachable verteces. If $G$ contains a spanning tree rooted at $v$, then
$T$ is also a spanning tree. 

We now turn to DFS\@. DFS or Depth First Search uses a stack, rather than a
queue. If $T$ is a from a DFS on graph $G$ starting from vertex $v$. 

We now turn to Dijkstra's Algorithm. 

\emph{By the end of the week (tomorrow) send in your choice of 3 papers to
professor Isaacs.} 

Now, we skip to chapter 1.5.5. The distributed versions of these algorithms. Now
we turn to a distributed BFS algorithm, or the flooding algorithm. Let $S$ be a
network containing a spanning tree rooted at $v$. Each child knows its parent
and root knows it has no parent. However, each node may not know it's children. 
Source broadcasts the token to its out-neighbors. In each communication round,
each node determines if it heard a token from its in-neighbors. If it gets a
token, the node stores the token in ``data'' the node stores the id of one of
the transmitting in-neighbors as a ``parent.'' In the subsequent round, this
node will broadcast token to out-neighbors. 

Next week we will cover Distributed Bellman Ford. Also read what's in the book
for flooding algorithms. 




\end{document}
